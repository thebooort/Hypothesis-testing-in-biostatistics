\documentclass[a4paper,12pt]{article}
\usepackage[utf8]{inputenc}
\usepackage[spanish]{babel}
\usepackage{amsthm}
\usepackage{amsmath}
\usepackage{amsfonts}
\usepackage{pgf,tikz}
\usepackage{graphicx}
\usepackage{float}
\usepackage{cite}
\usepackage{enumerate}
\usepackage{courier}
\usepackage{listings}
\usepackage{color}

\definecolor{codegreen}{rgb}{0,0.6,0}
\definecolor{codegray}{rgb}{0.5,0.5,0.5}
\definecolor{codepurple}{rgb}{0.58,0,0.82}
\definecolor{backcolour}{rgb}{0.95,0.95,0.92}

\lstdefinestyle{mystyle}{
	backgroundcolor=\color{backcolour},   
	commentstyle=\color{codegreen},
	keywordstyle=\color{magenta},
	numberstyle=\tiny\color{codegray},
	stringstyle=\color{codepurple},
	basicstyle=\footnotesize\ttfamily,
	breakatwhitespace=false,         
	breaklines=true,                 
	captionpos=b,                    
	keepspaces=true,                 
	numbers=left,                    
	numbersep=5pt,                  
	showspaces=false,                
	showstringspaces=false,
	showtabs=false,                  
	tabsize=2
}

\lstset{style=mystyle}
\usepackage{hyperref}
%\graphicspath{ {-} }
\usepackage[toc,page]{appendix}

%opening
\title{Hypothesis testing in biostatistics}
\author{Ruhugu, cocoseva, thebooort}

\begin{document}

\maketitle

\begin{abstract}
We do science á.Just for books to appear in our bibliography for now:
\cite{velez1993principios} \cite{rosner2015fundamentals}
\end{abstract}

\section{Summary and utilities del \cite{velez1993principios} }
\subsection{Tema 7}
\subsubsection{metodo de los momentos:}
Igualar los primeros momentos teóricos poblacionales (aquellos no constantes) a los correspondientes momentos muestrales hasta obtener un sistema de ecuaciones resoluble:
$$E_\theta[X^r]=\alpha(\theta_1,...)$$
$$a_r=\frac{1}{n}\sum_{i=1}^{n}X_i^r$$
$$\alpha_r=a_r $$
Aunque pueden funcionar carecen de justificación seria.
\subsubsection{metodo maxima verosimilitud}
Dada una m.a.s. de un apoblacion, la aproximacion de los parametros debe hacerse como aquella que maximiza la probabilidad de obtener esa m.a.s.
Special guest: ecuaciones de verosimilitud:
$$\frac{\partial}{\partial\theta_j}log f_\theta(x_1,x_2,...)=0$$
$$\forall j$$
\subsubsection{propiedades asintoticas de los estimadores de maxima vero1similitud}
Quien dice esto dice cascar dos teoremas. No creo que sean main point del trabajo, pero se pueden mirar
\subsubsection{estimacion bayesiana}
Esto si que creo que no entra nada, pasando.
\subsubsection{estimacion minimo cuadratica}
Dados los vectores de prediccion y experimentacion, ajustarlos de manera que la norma eculidea entre ellos sea la minima. 
\subsubsection{Ejercicios con aspectos teoricos pero no hay ej. signi de lo que buscamos}
\subsection{Tema 8}












\bibliographystyle{alpha}
\bibliography{bibliography/bibliografia}

\end{document}
